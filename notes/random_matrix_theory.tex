% Created 2025-04-01 Tue 11:28
% Intended LaTeX compiler: lualatex
\documentclass[11pt]{article}
\usepackage{amsmath}
\usepackage{fontspec}
\usepackage{graphicx}
\usepackage{longtable}
\usepackage{wrapfig}
\usepackage{rotating}
\usepackage[normalem]{ulem}
\usepackage{capt-of}
\usepackage{hyperref}
\input{../.latex/packages}
\input{../.latex/set}
\input{../.latex/commands}
\date{}
\title{Random Matrix Theory}
\hypersetup{
 pdfauthor={},
 pdftitle={Random Matrix Theory},
 pdfkeywords={},
 pdfsubject={},
 pdfcreator={Emacs 30.1 (Org mode 9.7.26)}, 
 pdflang={English}}
\begin{document}

{\centering
{\LARGE Random Matrix Theory \par }}
\section*{April 1, 2025}
\label{sec:org0bcd32d}
\subsection*{Preliminaries}
\label{sec:orgafec138}
Let \(\xi_{ij}\), \(\eta_{ij}\) be normal random variables (i.e. Gaussian, mean 0, variance 1).\\
e.g. \(\PP(\xi_{11}<s)=\int_{-\infty}^{s}\frac{1}{\sqrt{2\pi}}e^{-x^{2}/2}\;dx\).\\
\(\int_{-\infty}^{\infty}x^{2}\cdot\frac{1}{\sqrt{2\pi}}e^{-x^{2}/2}\;dx\) is the variance.\\
\(\frac{1}{\sqrt{2\pi}}e^{-x^{2}/2}\) is the Probability Density Function (PDF).\\
\(\frac{1}{\sqrt{2\pi}}e^{-x^{2}/2}\;dx\) is the probability measure on our probability space (i.e. totally finite measure space).\\
We build matrices\\
\begin{align*}
  \begin{bmatrix}
    \xi_{11} & \frac{\xi_{12}+i\eta_{12}}{\sqrt{2}} & \frac{\xi_{13}+i\eta_{13}}{\sqrt{2}} & \cdots \\
    \frac{\xi_{21}+i\eta_{21}}{\sqrt{2}} & \xi_{22} & \frac{\xi_{22}+i\eta_{22}}{\sqrt{2}} \\
    \frac{\xi_{31}+i\eta_{31}}{\sqrt{2}} & \frac{\xi_{32}+i\eta_{32}}{\sqrt{2}} &  \xi_{33} \\
    \vdots & & & \ddots
  \end{bmatrix}
\end{align*}
\subsubsection*{Homework}
\label{sec:org9e734df}
Is the PDF of \(\frac{a+b}{2}\) the same as \(\frac{\xi_{12}}{\sqrt{2}}\) for normal RVs \(a,b,\xi_{12}\)?\\
i.e. \(\PP\left( \frac{a+b}{2}<s) \right)\overset{?}{=}\left( \PP\frac{\xi_{12}}{\sqrt{2}} \right)\)\\
\subsection*{2x2 Random Matrix}
\label{sec:orgc31868d}
Our matrix \(L\) corresponds to eigvenvalues \(\lambda_{1},\lambda_{2}\) which are random variables determined by \(\{\xi_{ij},\eta_{ij}\}\).\\
Then the number of evaulations in the interval \(B\) is given by \(\sum_{j=1}^{2}\chi_{B}(\lambda_{j})\). We may take the average by\\
\begin{align*}
  \int_{-\infty}^{\infty}\cdots\int_{-\infty}^{\infty}\sum_{j=1}^{2}\chi_{B}(\lambda_{j})\frac{1}{\sqrt{2\pi}}e^{-\xi^{2}_{11}}\cdot\frac{1}{\sqrt{2\pi}}e^{-\xi^{2}_{22}}\cdot\frac{1}{\sqrt{2\pi}}e^{-\xi^{2}_{12}}\cdot\frac{1}{\sqrt{2\pi}}e^{-\eta^{2}_{12}}\;d\xi_{11}d\xi_{22}d\xi_{12}d\eta_{12}.
\end{align*}
\subsection*{Expected Evaluations}
\label{sec:org92bcc73}
We have that the expecation of the number of evaluations in the interval \((a,b)\) is given by \(\int_{a}^{b}G(s)\;ds\) where\\
\begin{align*}
  G(s)
  =e^{-\frac{s^{2}}{2}}\sum_{\ell=0}^{2}P_{\ell}(s)^{2}
\end{align*}
and \(P_{\ell}(s)\) is the Hermite polynomial of degree \(d\).\\
\end{document}

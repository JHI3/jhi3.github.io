% Created 2025-03-31 Mon 15:03
% Intended LaTeX compiler: lualatex
\documentclass[11pt]{article}
\usepackage{amsmath}
\usepackage{fontspec}
\usepackage{graphicx}
\usepackage{longtable}
\usepackage{wrapfig}
\usepackage{rotating}
\usepackage[normalem]{ulem}
\usepackage{capt-of}
\usepackage{hyperref}
\input{../.latex/packages}
\input{../.latex/set}
\input{../.latex/commands}
\date{}
\title{Manifolds III}
\hypersetup{
 pdfauthor={},
 pdftitle={Manifolds III},
 pdfkeywords={},
 pdfsubject={},
 pdfcreator={Emacs 30.1 (Org mode 9.7.26)}, 
 pdflang={English}}
\begin{document}

{\centering
{\LARGE Manifolds III \par }}
\section*{March 31, 2025}
\label{sec:org5a0a6c1}
\section*{Review}
\label{sec:orgd37a1b1}
If \(X,Y\) are topological spaces and \(f,g:X\to Y\) continuous maps, we say \(f\) and \(g\) are homotopic (written \(f\simeq g\)) if there is a homotopy \(H:X\times I\to Y\) (where \(I=[0,1]\)) such that \(H(x,0)=f(x)\) and \(H(x,1)=g(x)\) for all \(x\in X\).\\
We say that \(f\) is null-homotopic if it is homotopic to a constant map.\\
\subsection*{Proposition}
\label{sec:org04f8ec6}
Homotopy is an equivalence relation on the collection of continuous maps.\\
\begin{enumerate}
\item \(f\simeq f\) by \(H(x,t):=f(x)\).\\
\item \(f\overset{\tilde{H}}\simeq g\implies g\simeq f\) by defining \(\tilde{H}(x,t):=H(x,1-t)\).\\
\item \((f\overset{F}\simeq g\land g\overset{G}\simeq h)\implies f\simeq h\) by\\
\end{enumerate}
\begin{align*}
  H(x,t):=\begin{cases}
    F(x,2t) & 0\leq t\leq 1/2 \\
    G(x,2t-1) & 1/2\leq t\leq 1
  \end{cases}.
\end{align*}
\subsection*{Proposition}
\label{sec:org1933e99}
For \(f_{0},f_{1}:X\to Y\) and \(g_{0},g_{1}:Y\to Z\), if \(f_{0}\overset{F}\simeq f_{1}\) and \(g_{0}\overset{G}\simeq g_{1}\), then \(g_{0}\circ f_{0}\simeq g_{1}\circ f_{1}\).\\
\subsubsection*{Proof}
\label{sec:orgd98e8a7}
Define \(H(x,t):=G(F(x,t),t)\) such that \(H(x,0)=G(F(x,0),0)=G(f_{0}(x),0)=g_{0}\circ f_{0}(x)\).\\
Similarly, \(H(x,1)=g_{1}\circ f_{1}(x)\).\\
\subsection*{Definition: Homotopic Spaces}
\label{sec:orgd66ca0c}
We say that two spaces \(X\) and \(Y\) are homotopic to each other (\(X\simeq Y\)) if there are continuous maps \(f:X\to Y\) and \(g:Y\to X\) such that \(f\circ g\simeq\operatorname{id}_{Y}\) and \(g\circ f\simeq\operatorname{id}_{X}\).\\
\subsubsection*{Example}
\label{sec:org2d096de}
\(\R^{n}\) is homotopic to \(\{0\}\) (or any single point) by \(\iota:0\to\R^{n}\) and \(r:\R^{n}\to 0\). Then \(r\circ\iota:0\to 0\) is \(\operatorname{id}_{0}\) and \(\iota\circ r:\R^{n}\ni x\mapsto 0\in \R^{n}\) is homotopic to \(\operatorname{id}_{\R^{n}}\). In fact, consider \(H:\R^{n}\times I\to\R^{n}\) where \(H(x,t)=tx\), \(H(x,1)=x=\operatorname{id}_{\R^{n}}(x)\) and \(H(x,0)=0\).\\
\subsection*{Definition: Path}
\label{sec:org7d41851}
A path in \(X\) from \(p\) to \(q\) is a continuous map \(f:I\to X\) such that \(f(0)=p\) and \(f(1)=q\).\\
\subsection*{Definition: Path Homotopic}
\label{sec:org5527582}
Let \(f,g:I\to X\) be two paths in \(X\) from \(p\) to \(q\).\\
We say that \(f\) and \(g\) are path homotopic (write \(f\sim g\)) if there is a homotopy \(H:I\times I\to X\) such that \(H(s,0)=f(s)\), \(G(s,1)=g(s)\), \(H(0,t)=p\) and \(H(1,t)=q\).\\
\subsection*{Proposition}
\label{sec:orgb2ce404}
Path homotopy is an equivalence relation on the collection of paths from \(p\) to \(q\).\\
Write \([f]\), the equivalence class of \(f\) in the quotient.\\
\subsection*{Definition: Loop}
\label{sec:org818363b}
In the special case that \(p=q\), we say that \(f:I\to X\) is a loop\\
\section*{Definition: Fundamental Group}
\label{sec:org6da0872}
Given \((X,p)\), \(\pi_{1}(X,p)\) (the fundamental group of \(X\) at the point \(p\)) is the set of all loops at \(p\) modulo the path homotopy.\\
\begin{align*}
  \{\text{loops at }p\}/\sim
\end{align*}
Equivalently, \((S^{1},1)\), \(\{\text{loops at }p\}=\{\text{continuous maps }f:(S^{1},1)\to(X,p)\}\) with \(f(1)=p\). We say this is the homotopy ``relative to \(1\in S^{1}\)''. We have \(H:S^{1}\times I\to X\) such that \(H(s,0)=f(s)\), \(H(s,1)=g(s)\) and \(H(1,t)=p\).\\
\section*{Definition: Free Homotopy}
\label{sec:org7b80a82}
For two loops \(f,g: S^{1}\to X\), we say that \(f\) and \(g\) are free homotopic if \(f\simeq g\).\\
\section*{Lemma}
\label{sec:org77f2f50}
When \(f:I\to X\) is a path from \(p\) to \(q\), if \(f\circ\varphi\) is a reparameterization of \(f\) then \((f\circ\varphi)\sim f\) where \(\varphi:I\to I\) satisfies \(\varphi(0)=0\) and \(\varphi(1)=1\).\\
\subsection*{Proof}
\label{sec:org349ab46}
Note that \(\varphi\) is homotopic to the identity map \(\operatorname{id}_{I}\) through \(H(s,t)=ts+(1-t)\varphi(s)\) since \(H(s,0)=\varphi(s)\) and \(H(s,1)=s=\operatorname{id}_{I}(s)\).\\
Then consider \(f\circ H:I\times I\to X\) which is a path homotopy between \(f\) and \(f\circ\varphi\).\\
\section*{Fundamental Group}
\label{sec:orga0001a6}
Let \(f,g:I\to X\) be two paths with \(f(1)=g(0)\).\\
Then we can ``compose'' (concatenate) \(f\) and \(g\) together \((f\cdot g):I\to X\) by\\
\begin{align*}
  (f\cdot g)(s):=\begin{cases}
    f(2s) & 0\leq s\leq 1/2 \\
    g(2s-1 & 1/2\leq s\leq 1
  \end{cases}.
\end{align*}
\subsection*{Lemma}
\label{sec:org5aff9e3}
If \(f_{0}\overset{F}\sim f_{1}\), \(g_{0}\overset{G}\sim g_{1}\) and \(f_{0}(1)=f_{1}(1)=g_{0}(0)=g_{1}(0)\), then \(f_{0}\cdot g_{0}\sim f_{1}\cdot g_{1}\).\\
\subsubsection*{Proof}
\label{sec:org15a7d4f}
Define\\
\begin{align*}
  H(s,t):=\begin{cases}
    F(2s,t) & 0\leq s\leq 1/2 \\
    G(2s-1,t) & 1/2\leq s\leq 1
  \end{cases}.
\end{align*}
Then\\
\begin{align*}
  H(s,0)=\begin{cases}
    F(2s,0)=f_{0}(2s) & 0\leq s\leq 1/2 \\
    G(2s-1,0=g_{0}(2s-1) & 1/2\leq s\leq 1
  \end{cases}.
\end{align*}
Similarly \(H(s,1)=(f_{1}\cdot g_{1})(s)\), hence \(f_{0}\cdot g_{0}\sim f_{1}\cdot g_{1}\).\\
With this, we have a well-defined \([f]\cdot[g]:=[f\cdot g]\).\\
\subsection*{Simple Properties}
\label{sec:org54fb9e2}
For \(f\) from \(p\) to \(q\) where \(c_{p}\) is the constant map at \(p\),\\
\begin{enumerate}
\item \([c_{p}]\cdot[f]=[f]=[f]\cdot[c_{q}]\) since \(c_{p}\cdot f\) is a reparameterization of \(f\).\\
\item Let \(\overline{f}\) be the inverse path of \(f\) (i.e. \(\overline{f}(s)=f(1-s)\)). Then \([f]\cdot[\overline{f}]=[c_{p}]\) and \([\overline{f}]\cdot[f]=[c_{q}]\).\\
\end{enumerate}
\begin{align*}
  H(s,t):=\begin{cases}
    f(2s) & 0\leq s\leq t/2 \\
    f(t) & t/2\leq s\leq 1-t/2 \\
    f(2-2s) & 1-t/2 \leq s\leq 2
  \end{cases}.
\end{align*}
\begin{enumerate}
\item \(([f]\cdot[g])\cdot[h]=[f]\cdot([g]\cdot[h])\), since these are reparameterizations of the same path.\\
\end{enumerate}
\subsection*{Group Structure}
\label{sec:org5a26106}
\(\pi_{1}(X,p)=\{\text{loops at }p\}/\sim\).\\
Define \([f]\cdot[g]:=[f\cdot g]\).\\
It has an identity element \([c_{p}]=e\).\\
For any \(f\in\pi_{1}(X,p)\), it has an inverse \([\overline{f}]\) such that \([f]\cdot[\overline{f}]=[\overline{f}]\cdot[f]=[c_{p}]\).\\
Finally, it is associative by (3) above.\\
\subsection*{Proposition}
\label{sec:orgbf04093}
Suppose \(p,q\in X\) with \(X\) path-connected.\\
Then \(\pi_{1}(X,p)\) is isomorphic to \(\pi_{1}(X,q)\).\\
Remark: this isomorphism is not canonical.\\
\subsubsection*{Proof}
\label{sec:org1b0caa1}
We define a path \(\gamma\) from \(q\) to \(p\) and \(\Phi_{\gamma}:\pi_{1}(X,p)\to\pi_{1}(X,q)\) by \([f]\mapsto[\gamma\cdot f\cdot \overline{\gamma}]\).\\
\(\Phi_{\gamma}\) is a group homomorphism.\\
\begin{align*}
  \Phi_{\gamma}[f]\cdot\Phi_{\gamma}[g]
  &=[\gamma\cdot f\cdot\overline{\gamma}]\cdot[\gamma\cdot g\cdot \overline{\gamma}] \\
  &=[\gamma\cdot f\cdot \overline{\gamma}\cdot\gamma \cdot g\cdot \overline{\gamma}] \\
  &=[\gamma\cdot f]\cdot\overbrace{[\overline{\gamma}\cdot\gamma]}^{=e}\cdot[g\cdot \overline{\gamma}] \\
  &=[\gamma\cdot(f\cdot g)\cdot\overline{\gamma}] \\
  &=\Phi_{\gamma}[f\cdot g].
\end{align*}
\(\Phi_{\gamma}\) has an inverse, \(\Phi_{\overline{\gamma}}:\pi_{1}(X,q)\to\pi_{1}(X,p)\).\\
\begin{align*}
  \Phi_{\overline{\gamma}}\circ\Phi_{\gamma}[f]
  =\Phi_{\overline{\gamma}}[\gamma\cdot f\cdot\overline{\gamma}]
  =[\overline{\gamma}\cdot\gamma\cdot f\cdot \overline{\gamma}\cdot\gamma]
  =[f].
\end{align*}
\subsection*{Induced Homomorphism}
\label{sec:orgc0a1d65}
\(\varphi:(X,p)\to(Y,q)\) induces\\
\begin{align*}
  \varphi_{*}:\pi_{1}(X,p) &\to\pi_{1}(Y,q) \\
  [f]&\mapsto[\varphi\circ f].
\end{align*}
\(\varphi_{*}\) is a homomorphism.\\
\begin{align*}
  (\varphi_{*}[f])\cdot(\varphi_{*}[g])
  =[\varphi\circ f]\cdot[\varphi\circ g]
  =[(\varphi\circ f)\cdot(\varphi\circ g)]
  =[\varphi\circ(f\cdot g)]
  =\varphi_{*}[f\cdot g].
\end{align*}
\subsection*{Proposition}
\label{sec:orga4b8288}
If \(\varphi,\psi:(X,p)\to(Y,q)\) are homotopic, then \(\varphi_{*}=\psi_{*}:\pi_{1}(X,p)\to\pi_{1}(Y,q)\).\\
\subsubsection*{Proof}
\label{sec:org7c8a56b}
Let \([f]\in\pi_{1}(X,p)\), \(\varphi_{*}[f]=[\varphi\circ f]\) and \(\psi_{*}[f]=[\psi\circ f]\) and \(H:X\times I\to Y\) a homotopy between \(\varphi\) and \(\psi\).\\
Then define \(\tilde{H}:=I\times I\to Y\) by \(\tilde{H}(s,t)=H(f(s),t)\) such that\\
\begin{align*}
  \tilde{H}(s,0)&=H(f(s),0)=\varphi\circ f(s) \\
  \tilde{H}(s,1)&=H(f(s),1)=\psi\circ f(s).
\end{align*}
\subsection*{Corollary}
\label{sec:org042564b}
If \(X\simeq Y\), then \(\pi_{1}(X)\simeq\pi_{1}(Y)\).\\
\subsection*{Examples}
\label{sec:orgbe87c89}
\(\pi_{1}(S^{1})\cong\Z\) and \(\pi_{1}(S^{n})=0\) for \(n\geq2\).\\
For \(n\geq 2\), write \(S^{n}=A_{+}\cup A_{-}\) where \(A_{+}\) and \(A_{-}\) are large balls centered at the north and south pole respectively.\\
Then \(A_{+}\) and \(A_{-}\) are both homeomorphic to \(\R^{n}\) and \(A_{+}\cap A_{-}\) (their intersection about the equator) is homemorphic to \(S^{n-1}\times\R\).\\
We fix a base point \(p\in A_{+}\cap A_{-}\) and let \(f:I\to S^{n}\) be a loop based at \(p\).\\
There exists a partition of \(I\), \(0=s_{0}<s_{1}<\cdots<s_{k}=1\), such that \(f|_{[s_{i},s_{i+1}]}\) is contained in \(A_{-}\) or \(A_{+}\).\\
Draw a path \(\gamma_{i}\) from \(p\) to \(f(s_{i})\) such that \(\gamma_{i}\subseteq A_{+}\cap A_{-}\). Let \(f_{i}=f|_{[s_{i},s_{i+1}]}\) such that \(f=f_{0}\cdot f_{1}\cdots f_{k}\). Then this is path homotopic to\\
\begin{align*}
  (f_{0}\cdot\overline{\gamma}_{1})\cdot(\gamma_{1}\cdot f\cdot\overline{\gamma}_{2})\cdots(\gamma_{k-1}\cdot f_{k-1}\cdot\overline{\gamma}_{k})\cdot(\gamma_{k}\cdot f_{k}).
\end{align*}
Each \(\gamma_{i}\cdot f_{i}\cdot\overline{\gamma}_{i}\) is contained in \(A_{-}\) or \(A_{+}\), hence \(\gamma_{i}\cdot f_{i}\overline{\gamma}_{i+1}\sim c_{p}\), \(f\simeq c_{p}\) and \([f]=e\).\\
\end{document}
